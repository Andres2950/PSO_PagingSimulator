% !TeX spellcheck = es
\documentclass{report}
\usepackage[utf8]{inputenc}

% Títulos automáticos en español
\usepackage[spanish]{babel}

% Soporte para buenas urls e hipervínculos entre secciones
\usepackage{hyperref}

% Citas y referencias en formato APA
% Si quiere las citas y referencias en IEEE comente esta línea
\usepackage{apacite}

% Imágenes y figuras
\usepackage{graphicx}

% Código fuente con números de línea
\usepackage{listings}
% Puede cambiar el lenguaje de código fuente
% https://www.overleaf.com/learn/latex/code_listing#Supported_languages
\usepackage{multirow}


\lstset{
    language=C,
    basicstyle=\footnotesize,
    numbers=left,
    stepnumber=1,
    showstringspaces=false,
    tabsize=1,
    breaklines=true,
    breakatwhitespace=false,
}


\def \unidad{Escuela de Ingeniería en Computación}
\def \programa{Ingeniería en Computación}
\def \curso{IC6600 - Principios de Sistemas Operativos}
\def \titulo{Proyecto 2}
\def \subtitulo {Simulación de Algoritmos de Paginación}
\def \autores{
    Gerald Calderón\\
    gecalderon@estudiantec.cr\\
    2023125197\\
    
    \vspace{0.5cm}
    
    Óscar Obando\\
    osobando@estudiantec.cr\\
    2023091684
    
    \vspace{0.5cm}
    
    Samuel Zúñiga\\
    sazuniga@estudiantec.cr	\\
    2023029693
}
\def \fecha{Octubre 2025}
\def \lugar{
    San José, 
    Costa Rica
}

% Inicia el documento 
\begin{document}

% Inserta la portada del documento
\input{portada}

\tableofcontents

\chapter{Introducción}\label{intro}

Para este proyecto se implementó un compresor (y decompresor) de archivos de texto utilizando el algoritmo de Huffman.
Este toma un directorio con archivos de texto y los comprime en un solo archivo en binario, al descomprimir, se restaura el directorio y los archivos dentro de este.
Este sistema consiste en dos partes principales,  el programa que comprime y el programa que descomprime.
Ambos ofrecen tres modos de ejecución: serial, concurrente y paralelo.

\section {Algoritmos de Paginación}
\subsection{Optimo}
\subsection{FIFO y Second Chance}
\subsection{LRU y MRU}
\subsection{Random}

\begin{figure}[h]
    \centering
    \includegraphics[width=0.8\linewidth]{figuras/estructura.png}
    \caption{Ilustración de la estructura del programa}
    \label{fig:estructura}
\end{figure}

\section{Instrucciones}
\subsubsection{Cómo instalar el programa}
\begin{enumerate}
  \item Descargue el código fuente del programa. Puede hacerlo de las dos siguientes formas:
    \begin{itemize}
      \item Dirigirse al repositorio de GitHub mediante su navegador a través del siguiente link: \url{https://github.com/Andres2950/SO\_Huffman.git}
      \item Instalarlo directamente con el comando \\
    \texttt{wget \url{https://github.com/Andres2950/SO\_Huffman/archive/refs/heads/main.zip}}
    \end{itemize}
  \item Descomprima el archivo .zip descargado utilizando el comando \texttt{unzip}.
  \item Al extraer el archivo podrá observar la estructura de organización similar a la figura \ref{fig:estructura}.
\item Ejecute el archivo \texttt{install.sh}, asegúrese de qué tenga permisos de ejecución, puede utilizar el comando \texttt{chmod +x install.sh} en caso de que no los posea y luego ejecute de la siguiente forma \texttt{./install.sh}. \\
  Este archivo se hará cargo de la instalación del compilador \textit{gcc} necesario para compilar el código fuente. Además hará la instalación del paquete \textit{make} para la ejecución de archivos makefile, dicho archivo posee las instrucciones de compilación de \textit{gcc} para resolver todas las dependencias que se encuentran en el directorio \textit{src/headers}.  
\item Note que la instalación de dichos paquetes requiere permisos de usuarios root, al ejecutar el archivo \texttt{install.sh} este se volverá a ejecutar con dichos permisos, para esto solicitará la contraseña del usuario root para tener dichos permisos de ejecución (la contraseña es totalmente invisible para el programa) y así poder descargar los paquetes. Mostrará un mensaje como el de la figura N y esperará la entrada correcta.
\item La compilación de los archivos se realiza sin permisos root.
\item Además, el \texttt{install.sh} hará una copia de los binarios \texttt{huff} y \texttt{dehuff} en el directorio \textit{/usr/bin} de la máquina para poner ser accedidos desde cualquier lugar y utilizado en múltiples directorios sin necesidad de mover los archivos a comprimir o descomprimir a una carpeta en particular.

\end{enumerate}

\subsection{Cómo utilizar el programa}
Una vez terminado el procedimiento de instalar el programa puede utilizar el comando huff para comprimir archivos. Dicho comando posee una serie de parámetros para sus distintos modos de ejecución. A continuación se detallan cada uno.

\begin{figure}[h]
    \centering
    \includegraphics[width=0.8\linewidth]{figuras/huff_ayuda.png}
    \caption{Ilustración del menú de ayuda del comando huff -h}
    \label{fig:huffayuda}
\end{figure}

\begin{itemize}
  \item \textbf{huff -h}\\ \hspace{2cm}
    Provee una guía de los parámetros requeridos por el programa y su orden de entrada (Véase figura \ref{fig:huffayuda}).
  \item \textbf{huff [Opción] src dst} o \textbf{dehuff [Opción] src dst}\\
src es el directorio objetivo a ser comprimido/descomprimido\\
dst es el archivo de destino
  \item \textbf{Opciones} 
    \begin{itemize}
      \item \textbf{-s} modo de ejecución serial.
      \item \textbf{-p} modo de ejecución paralela usando fork.
      \item \textbf{-c} modo de ejecución concurrente usando pthread
      \item \textbf{-b} modo de Benchmark para comparar los tiempos de ejecuión entre los anteriores modos.
    \end{itemize}
\end{itemize}

\subsubsection{Tabla de resultados}
\begin{center}
	\begin{tabular}{|c|c|c|c|}		
\hline
\multicolumn{4}{|c|}{Tiempo de Ejecución (milisegundos)} \\
\hline
 Subsistema& \multicolumn{3}{|c|}{Versión} \\
 \hline
 & Serial & Concurrente & Paralelo\\
 \hline
huff(completo) & 5360& 2226 & 2036\\
 \hline
huff(lectura) &  36 & 25 & 19\\
 \hline
 huff(compresión) &  5168 & 1957 & 1681\\
 \hline
 dehuff(completo) & 5189 & 2092 & 1855\\
 \hline
 dehuff(descompresión y escritura) & 5146 & 1968 & 1629\\
 \hline
	\end{tabular}
\end{center}

\section{Conclusiones}
Los resultados obtenidos muestran una clara mejora en los progamas de compresión y de descompresión, a partir de esto se puede concluir que el algoritmo de Huffman implementado aprovecha mucho de las ventajas que proveen la concurrencia y paralelización. 
Esto se debe a que ambos programas manejan distintos archivos independientes entre ellos.
Esta independencia permite que los archivos puedan ser manejados por separado, sea este procesamiento por medio de hilos o de procesos hijo.
Por lo que se puede decir que cuando hay muchas subtareas independientes por hacer en un programa, es bueno considerar alguno de los dos. 

A pesar de lo dicho anteriormente, ambos en ambos modos de ejecución, ninguno se acercó al límite teórico que se propuso para cada uno.
A partir de esto se puede concluir que el overhead que produce el uso de hilos o de subprocesos no es trivial.
Al usar cualquiera de estos dos siempre es importante tomar en cuenta el overhead que estos causan para determinar si realente vale la pena usar alguna de las técnicas descritas anterormente.

Comparando los resultados que dieron las pruebas de concurrencia y las pruebas de paralelización, se puede concluir que la paralelización es ligeramente mejor que la concurrencia.
Esto se puede deber a la naturaleza de la paralelización, gracias a que cada proceso puede correr en un procesador distinto. 
También se puede deber a la necesidad de la concurrencia de cambiar de contexto frecuentemente, lo que también toma recursos de la computadora.


% Estilo de bibliografía APA
% Si quiere usar el estilo IEEE comente esta línea
\bibliographystyle{apacite}

% Descomente esta línea para usar el estilo de bibliografía IEEE
%\bibliographystyle{ieeetr}
\bibliography{referencias}

\end{document}
